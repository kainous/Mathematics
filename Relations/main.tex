\begin{part}{Relations}
    \begin{chapter}{Binary Relations}
        \begin{section}{Heterogenous and Homogeneous Relations}
            Homogeneous means same-type, and heterogenous means different-type. In this regard, 
            
            Oftentimes, mathematically, we don't have specific words that mean possibly one and possibly the other. This is important, because a heterogenous relation can be homogenous too. The phrase non-homogenous specifically refers to a relation that must have different types.
        \end{section}
        
        \begin{section}{Heytingness}
            Make binary relation as ``included subsets'' ($R \in U\times V$) to a functional predicate for inclusion ($R : U \times V \to \mathbb{B}$) and that corresponds to ($R:U\to V \to \mathbb{B}$).
        \end{section}
        
        \begin{section}{asdf}
        \end{section}
    \end{chapter}
    
    \begin{chapter}{n-ary Relations}
    
    \end{chapter}
    
    \begin{chapter}{Relation Algebra}
    
    \end{chapter}
    
    \begin{chapter}{Other}
        A filter is a unary relation (can be useful as base-case in inductive arguments)
        
        A unary relation is a filter on a set (a subset)
        
        There are only 2 nullary relations: always holding and never holding
        
        A binary relation is a subset of the cartesian product
        
        A ternary relation requires an ordered n-tuple, since cartesian products are not associative. However, ((a, b), c) is also valid for discussion.
        
        
    \end{chapter}
\end{part}  