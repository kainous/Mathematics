\chapter{Logical Deduction}
\section{Classical Logic}
Socrates tells us that Plato gave us 3 \gls{LawsOfThought} (paraphrased here, and using the notion of ``true'' as a value):
\begin{itemize}
\item \textbf{Law of Identity} –- For any statement P, P = P. This is to say that true is true, false is false, and this law basically defines a notion of equality for logic.
\item \textbf{Law of Non-Contradiction} –- No two contradictory statements can both be true. For instance, it cannot be that ``Humans are animals'' and ``Humans are not animals''.
\item \textbf{Law of the Excluded Middle} –- Everything must be either true or false.
\end{itemize}

There are also records of the Law of Non-Contradiction in the Indian Sutras, but the most well-known collection of these 3 laws is by Socrates.
Aristotle also gave us one more law, codified as the \textbf{syllogism}:
If we have the following two statements:
\begin{itemize}
\item If X then Y
\item X
\end{itemize}
Then we can conclude Y; and hence, justification in intuitionistic logic is interpreted as “reachability”, the ability to reach Y from X.

An example of a syllogism would be, “If it is raining in the open field, then the ground is getting wetter”, and “it is raining” giving us the result that “the ground is getting wetter”

Now, it may seem counterintuitive, but we cannot conclude the logical converse (swapping). In other words, we cannot use wet ground to conclude that it is raining, because sprinklers may be on, or the water hose running over the ground.

\todo{treat it like a path}